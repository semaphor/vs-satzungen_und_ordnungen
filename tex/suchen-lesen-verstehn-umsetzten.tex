% !TeX root = ../stuve-handbuch-latex.tex
% !TeX encoding = UTF-8
% !TeX spellcheck = de_DE
% !TeX program = pdflatex

%\newgeometry{left=2.2cm, right=2cm, top=1cm, bottom=1cm}


% % % % ALTERNATIVE ;-)
% wurde aber rausgelassen, da evtl. unverständlich und demotivierend

%\vfill
%
%\begin{figure}[h]
%	\centering
%	\hspace{1.5cm}\includegraphics[width = 0.6\textwidth, keepaspectratio]
%	{./grafiken/suchen-lesen-verstehen-umsetzen.jpg}
%	%	{./grafiken/suchen-lesen-verstehen-umsetzen-original-cut.jpg}
%\end{figure}
%
%\begin{minipage}{0.8\textwidth}
%\textbf{Verbitterte alte Männer}\\\textit{2014, Filzstift auf Flipchartpapier, Text: Stefan T. Kaufmann, Umrahmung: Tobias Scheinert, Digitalisat: Simon B. Lüke %(dabei später Umrahmung entfernt.)}
%\end{minipage}
%
%\vfill


\null
\clearpage
\null
\vfill
%\vspace{2cm}
Zu unglaublich vielen Themen der StuVe wurde \textit{irgendwann früher}{\textsuperscript\texttrademark} schonmal \textit{irgendwas}{\textsuperscript\textcopyright} gemacht, gelöst, konzipiert, überlegt und oft auch aufgeschrieben. Vielleicht gibt es nur Informationen darüber, wer sich mal damit befasst hatte und ein paar alte Sitzungsprotokolle. Oder es gibt schon eine ausführliche Aufdröselung des Problems und mögliche Ideen zu Lösung. Bei einigen Vorgängen gibt es aber auch sehr ausführliche und ausgetüftelte Anleitungen.

Also muss man bei vielen Themen nicht von ganz vorne beginnen und es lohnt sich, sich erstmal umzuschauen -- z.\,B.~indem man kurz bei denen nachfragt, die schon ein paar Jahre dabei sind. Auf jeden Fall lohnt es sich aber in der Dateiablage und vor allem \textbf{im StuVe-Wiki zu suchen}. Das StuVe-Wiki ist nicht perfekt organisiert oder sofort zu durchschauen\footnote{Außerdem ist es oft ungünstig, dass  manche Informationen über mehrere verschiedenen Wikis verteilt wurden.}, aber das macht nichts denn es gibt eine Suchfunktion, mit der man schnell zu einem Ergebnis kommt\footnote{Nicht vergessen nicht nur nach den Seitenüberschriften, sondern auch im „Text“ suchen zu lassen ;-)}.

Nutzt diese Möglichkeiten und erspart euch viel unnötige Arbeit. Findet Anknüpfungspunkte, findet heraus welche anderen Stellen (in der StuVe, an der Uni oder sonstwo) schon Erfahrungen oder gar Kompetenzen in dem Bereich, in dem ihr was erreichen wollt, haben.
\vfill
\vfill


%\restoregeometry