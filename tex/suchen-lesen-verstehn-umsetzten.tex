% !TeX root = ../stuve-handbuch-1.tex
% !TeX encoding = UTF-8
% !TeX spellcheck = de_DE
% !TeX program = pdflatex


% % % % ALTERNATIVE ;-)
% wurde aber rausgelassen, da evtl. unverständlich und demotivierend

%\vfill
%
%\begin{figure}[h]
%	\centering
%	\hspace{1.5cm}\includegraphics[width = 0.6\textwidth, keepaspectratio]
%	{./grafiken/suchen-lesen-verstehen-umsetzen.jpg}
%	%	{./grafiken/suchen-lesen-verstehen-umsetzen-original-cut.jpg}
%\end{figure}
%
%\begin{minipage}{0.8\textwidth}
%\textbf{Verbitterte alte Männer}\\\textit{2014, Filzstift auf Flipchartpapier, Text: Stefan T. Kaufmann, Umrahmung: Tobias Scheinert, Digitalisat: Simon B. Lüke %(dabei später Umrahmung entfernt.)}
%\end{minipage}
%
%\vfill


\null
\clearpage

\addcontentsline{toc}{chapter}{Allgemeiner Hinweis}

\null
\vfill
%\vspace{2cm}
Zu unglaublich vielen Themen der StuVe wurde \textit{irgendwann früher}{\textsuperscript\texttrademark} schonmal \textit{irgendwas}{\textsuperscript\textcopyright} gemacht, gelöst, konzipiert, überlegt und oft auch aufgeschrieben. Vielleicht gibt es nur Informationen darüber, wer sich mal damit befasst hatte und ein paar alte Sitzungsprotokolle. Oder es gibt schon eine ausführliche Aufdröselung des Problems und mögliche Ideen zu Lösung. Und bei einigen Themen gibt es auch schon sehr ausführliche und ausgetüftelte Anleitungen.

Also ist es bei vielen Themen nicht nötig von ganz vorne zu beginnen, sondern es lohnt, sich erstmal umzuschauen -- z.\,B.~indem man erstmal kurz bei denen nachfragt, die schon ein paar Jahre dabei sind. Auf jeden Fall lohnt es sich aber in der Dateiablage und vor allem \textbf{im StuVe-Wiki zu suchen}. Das Wiki ist nicht perfekt organisiert oder sofort zu durchschauen\footnote{Außerdem ist ungünstig, dass leider manche Informationen über mehrere verschiedenen Wikis verteilt wurden.}, aber das macht nichts denn es gibt eine Suchfunktion, mit der man schnell zu einem eventuell schon vorhandenen Ergebnis kommt\footnote{Im Wiki nicht nur nach den Seitenüberschriften („Tiel“), sondern auch im „Text“ suchen lassen!}.

Nutzt diese Möglichkeiten und erspart euch viel unnötige Arbeit. Findet Anknüpfungspunkte, findet heraus welche anderen Stellen (in der StuVe, an der Uni oder sonst wo) schon Erfahrungen oder gar Kompetenzen in dem Bereich, in dem ihr gerade was erreichen wollt, haben.
\vfill
\vfill
