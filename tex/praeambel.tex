\usepackage{perpage}
\MakePerPage{footnote}		%perpage package command: fußnoten


\title{StuVe-Handbuch}
\date{\today}
\author{et al.\email{stuve.kontakt@uni-ulm.de}}

\usepackage[utf8]{inputenc}
\usepackage[T1]{fontenc}					% für echte Umlaute und mehr…
\usepackage{lmodern}						% schönere Schrift, v.a in PDFs
\renewcommand{\familydefault}{\sfdefault}
\usepackage[ngerman]{babel}					% deutsche Sprachvariante (Inhaltsverzeichnis, …)

%\usepackage[left=2.5cm, right=1.7cm, top=1cm, bottom=1cm]{geometry}
\usepackage[left=2.2cm, right=2cm, top=1cm, bottom=1cm]{geometry}
%\usepackage{geometry}
\usepackage{layout}							% für \layout{} → Darstellung der aktuellen Ränder und Co.
\usepackage{totcount}						% Zählen der Gesamtzahl, z.B. von Kapiteln
\regtotcounter{chapter}

\usepackage[final]{pdfpages}
\usepackage{graphicx}


% % % % % % % % % % % % % % % % % % % % % % % % % %
% Überschriften und Inhaltsverzeichnis formatieren
% % % % % % % % % % % % % % % % % % % % % % % % % %

\addto\captionsngerman{\renewcommand{\chaptername}{Teil}}

\usepackage{titlesec}
\usepackage{fancyhdr}

\renewcommand*{\chaptermark}[1]{ \markboth{\thechapter: ##1}{} }%


%\titlespacing*{\chapter}{0pt}{}{20pt}
\titleformat{\chapter}[display]{\normalfont\huge\bfseries}{\flushright \chaptertitlename\ \thechapter}{5pt}{\flushright \Huge}

% Neues Kommando zum individuellen Einfügen des Inahltsverzeichnisses:
\makeatletter
\newcommand*{\toccontents}{\@starttoc{toc}}
\makeatother

\usepackage{titletoc}

%\titlecontents{⟨Abschnitt⟩}[⟨Links⟩]{⟨Gesamtformatierung⟩}{⟨Vor dem Label⟩}{⟨Nach dem Label⟩}{⟨Abstandsfüller & Seitenzahl⟩}[⟨Rechts⟩]
\titlecontents{chapter}[1.5em]{\small}{\contentslabel{1em}}{}{\titlerule*[0.3pc]{.}\contentspage}
\titlecontents{section}[2.5em]{\footnotesize}{\contentslabel{1em}}{}{\titlerule*[0.3pc].{}\contentspage}
%\titlecontents{subsection}[6.7em]{}{\contentslabel{2.95em}}{}{\titlerule*[0.3pc]{.}\contentspage} 
%\titlecontents{subsubsection}[10.6em]{}{\contentslabel{3.8em}}{}{\titlerule*[0.3pc]{.}\contentspage}
%\titlecontents{paragraph}[15.25em]{}{\contentslabel{4.6em}}{}{\titlerule*[0.3pc]{.}\contentspage}



% % % % % % % % % % % % %
% Abschnittsmarkierungen
% % % % % % % % % % % % %

\usepackage{tikz}
\usepackage{color}
\usetikzlibrary{shapes.misc}
\usetikzlibrary{calc}
\usetikzlibrary{positioning}
\usepackage{lipsum}

\usepackage{etoolbox}
% Patch the sectioning commands to provide a hook to be used later
\preto{\chapter}{\def\leveltitle{\chaptertitle}}
\preto{\section}{\def\leveltitle{\sectiontitle}}
\preto{\subsection}{\def\leveltitle{\subsectiontitle}}
\preto{\subsubsection}{\def\leveltitle{\subsubsectiontitle}}

\makeatletter
% \@sect is called with normal sectioning commands
% Argument #8 to \@sect is the title
% Thus \section{Title} will do \gdef\sectiontitle{Title}
\pretocmd{\@sect}
{\expandafter\gdef\leveltitle{#8}}
{}{}
% \@ssect is called with *-sectioning commands
% Argument #5 to \@ssect is the title
\pretocmd{\@ssect}
{\expandafter\gdef\leveltitle{#5}}
{}{}
% \@chapter is called by \chapter (without *)
% Argument #2 to \@chapter is the title
\pretocmd{\@chapter}
{\expandafter\gdef\leveltitle{#2}}
{}{}
% \@schapter is called with \chapter*
% Argument #1 to \@schapter is the title
\pretocmd{\@schapter}
{\expandafter\gdef\leveltitle{#1}}
{}{}
\makeatother

%\newcommand\MyColor{
%	\ifcase\thechapter blue!30\or red!30\or olive!30\or magenta!30\else yellow!30\fi} 
\newcommand\MyColor{ black!30 }

\fancypagestyle{plain}{%
	%% Clear all headers and footers
	\fancyhf{}
	%% Right headers on odd pages
	\fancyhead[RO]{%
			\begin{tikzpicture}[overlay,remember picture]
			\ifdefined\chaptertitle
			\node[
				fill=\MyColor, text=black,
				font=\footnotesize,
				inner ysep=4pt, inner xsep=4pt,
				rounded rectangle,
				anchor = west,
				%xshift=-0mm,
				%yshift=-0mm,
				%text width=3cm,
				text height=0.3cm,
				rotate=90
			]
			% 21 cm an der kurzen Kante der A4 bzw. langen Kante der A5 Seite:
			%   21 cm - 6 mm Druckrand oben - 6 mm Druckrand unten - 6 mm Länge Bobbel Seitenzahl = 19,2 cm
			at ($ (current page.north east) + (-0.9cm, -0.7cm) + (-0.15cm, -0.3cm) + (0cm, -19.3 / \totvalue{chapter} * \thechapter cm) $)
			{\sffamily\normalsize\nouppercase{\thechapter: \chaptertitle}};
%			{\sffamily\normalsize\nouppercase{\thechapter: }};
			\fi
			\node[
				fill=\MyColor,text=black,
				font=\footnotesize,
				inner ysep=4pt, inner xsep=4pt,
				rounded rectangle,
				%anchor = north east,
				%xshift=-0mm,
				%yshift=-0mm,
				%text width=5cm,
				text height=0.3cm,
				rotate=0
			]
			at ($ (current page.north east) + (-0.9cm, -0.7cm) + (-0.15cm, -0.15cm) $)
			{\sffamily\normalsize\nouppercase{\thepage}};
			\end{tikzpicture}
	}
	%% Left headers on even pages
	\fancyhead[LE]{%
			\begin{tikzpicture}[overlay,remember picture]
			\ifdefined\chaptertitle
			\node[
			fill=\MyColor,text=black,
			font=\footnotesize,
			inner ysep=4pt, inner xsep=4pt,
			rounded rectangle,
			anchor=east,
			%xshift=-0mm,
			%yshift=-0mm,
			%text width=3cm,
			text height=0.3cm,
			rotate=270
			]
			at ($ (current page.north west) + (0.9cm, -0.7cm) + (0.15cm, -0.3cm) + (0cm, -19.3 / \totvalue{chapter} * \thechapter cm) $)
			{\sffamily\normalsize\nouppercase{\thechapter: \chaptertitle}};
%			{\sffamily\normalsize\nouppercase{\thechapter:}};
			\fi
			\node[
			fill=\MyColor,text=black,
			font=\footnotesize,
			inner ysep=4pt, inner xsep=4pt,
			rounded rectangle,
			%anchor= ,
			%xshift=-0mm,
			%yshift=-0mm,
			%text width=5cm,
			text height=0.3cm,
			rotate=0
			]
			at ($ (current page.north west) + (0.9cm, -0.7cm) + (0.15cm, -0.15cm) $)
			{\sffamily\normalsize\nouppercase{\thepage} };
			\end{tikzpicture}
	}
	\renewcommand{\headrulewidth}{0pt}
	\renewcommand{\footrulewidth}{0pt}
	%\fancyfoot[R]{\thepage}
}
% ----------------------------------------------------------------


\usepackage{hyperref}

\setlength{\parindent}{0pt}        %  kein Einruecken bei Absaetzen 


\hypersetup{
	pdftitle=StuVe-Handbuch,
	pdfauthor=et. al.,
	pdfsubject={Handbuch},
	pdfproducer={pdflatex},
	%	colorlinks=false,
	pdfborder=0 0 0	% keine Box um die Links!
}

\newcommand{\leadingzero}[1]{\ifnum #1<10 0\the#1\else\the#1\fi} 