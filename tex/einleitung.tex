% !TeX root = ../stuve-handbuch-latex.tex
% !TeX encoding = UTF-8
% !TeX spellcheck = de_DE
% !TeX program = pdflatex


\textit{Definition: Die Verfasste Studierendenschaft der Universität Ulm (VS) setzt sich aus allen eingeschriebenen Studierenden zusammen. Die VS bzw. die Studierenden organisieren sich in der StuVe (StudierendenVertretung)\footnote{Diese Definition der StuVe fehlt noch in der Organisationssatzung, sollte demnächst ergänzt werden.}. D.h.\ alle Organe, Strukturen und auch sonst irgendwie offiziell aktiven Studierenden bilden die „organisierte“ StuVe – im Ggs. zur gesamten Studierendenschaft.}

\section*{Zu diesem Handbuch}

%\vspace{1em}
%\textbf{Zu diesem Handbuch}
%\vspace{0.3em}

Dieses Heft sammelt die wichtigsten Texte zur Arbeit in der StuVe. Es handelt sich nicht um ein am Stück geschriebenes Werk, sondern im Gegenteil um eine bloße Zusammenstellung verschiedener Texte. Einerseits sind einfache Hilfestellungen enthalten, wie z.B.\ die Infoblätter oder Grafiken, andererseits aber auch für alle Studierenden und insbesondere die gewählten Vertreter rechtlich verbindliche Texte, wie z.B.\ die Organisationssatzung oder die Finanzordnung, enthalten.

Da sich die Dokumente bisweilen ändern wird auch dieses Heft immer wieder aktualisiert werden, zum Vergleich dient das Datum auf der Titelseite. Manche der enthaltenen Dokumente haben eine eigene Seitennummerierung – im gesamten Heft sind jedoch am äußeren Rand jeweils die Bezeichnungen der Teile und durchgehende Seitennummern grau hinterlegt abgedruckt. Auf diese bezieht sich das folgende Inhaltsverzeichnis.

\subsection*{Inhalt}

%\vspace{1em}
%\textbf{Inhalt}
%\vspace{0.3em}

\toccontents

\subsubsection*{Nicht enthalten}
%\vspace{0.5em}
%\textbf{Nicht enthalten}
%\vspace{0.3em}
\begin{itemize}
	\item Wahlordnung
	\item Landeshoschulgesetz Baden-Württemberg (LHG)\\Informationen hierzu jedoch in Rechtliche Grundlagen und Rahmenbedingungen für die neue Studierendenvertretung.
\end{itemize}
