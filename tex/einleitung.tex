% !TeX root = ../stuve-handbuch-latex.tex
% !TeX encoding = UTF-8
% !TeX spellcheck = de_DE
% !TeX program = pdflatex


\textit{Definition: Die Verfasste Studierendenschaft der Universität Ulm (VS) setzt sich aus allen eingeschriebenen Studierenden zusammen.
Die VS bzw. die Studierenden organisieren sich in der StuVe (StudierendenVertretung)\footnote{\today: Diese Definition der StuVe fehlt noch in der Organisationssatzung, sollte demnächst ergänzt werden.}.
D.h.\ alle Organe, Strukturen und auch sonst irgendwie offiziell aktiven Studierenden bilden die „organisierte“ StuVe – im Ggs. zur gesamten Studierendenschaft.}


\section*{Zu diesem Heft}

%\vspace{1em}
%\textbf{Zu diesem Handbuch}
%\vspace{0.3em}

Für alle in der StuVe Aktiven und insbesondere die Mitglieder der StuVe-Gremien sind die hier die wichtigsten Texte zusammengefasst, die die Grundlage für die Organisation und Arbeit der StuVe bilden. Ergänzend zu diesem Handbuch soll demnächst noch ein zweites Handbuch mit Hilfen für's operative Tagesgeschäft verfügbar sein.
Es handelt sich hier nicht um ein am Stück geschriebenes Werk, sondern fast ausschließlich um eine Zusammenstellung verschiedener Texte.
Einerseits sind einfache Hilfestellungen, wie z.B.\ die Infoblätter oder Grafiken, andererseits aber auch für alle Studierenden und insbesondere die gewählten Vertreter und Aktiven rechtlich verbindliche Texte, wie z.B.\ die Organisationssatzung oder die Finanzordnung, enthalten.

Da sich die Dokumente bisweilen ändern wird auch dieses Heft immer wieder aktualisiert werden, zum Vergleich verschiedener Ausgaben dient das Datum auf der Titelseite.
Manche der enthaltenen Dokumente haben eine eigene Seitennummerierung – im gesamten Heft ist jedoch jeweils am äußeren Rand die Bezeichnungen des aktuellen Teils und eine durchgehende Seitennummern grau hinterlegt abgedruckt. Auf diese bezieht sich das folgende Inhaltsverzeichnis.


\subsection*{Inhalt}

%\vspace{1em}
%\textbf{Inhalt}
%\vspace{0.3em}

\toccontents


\subsubsection*{Nicht enthalten}
%\vspace{0.5em}
%\textbf{Nicht enthalten}
%\vspace{0.3em}
\begin{itemize}
	\item Wahlordnung
	\item Landeshoschulgesetz Baden-Württemberg (LHG)\\Informationen hierzu jedoch in Rechtliche Grundlagen und Rahmenbedingungen für die neue Studierendenvertretung.
\end{itemize}
