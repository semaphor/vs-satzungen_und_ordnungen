% !TeX root = ../stuve-handbuch-1.tex
% !TeX encoding = UTF-8
% !TeX spellcheck = de_DE
% !TeX program = pdflatex


\textit{Definition: Die Verfasste Studierendenschaft der Universität Ulm (VS) setzt sich aus allen eingeschriebenen Studierenden zusammen.
Die VS bzw. die Studierenden organisieren sich in der StuVe (StudierendenVertretung)\footnote{\today: Diese Definition der StuVe fehlt noch in der Organisationssatzung, sollte demnächst ergänzt werden.}.
D.h.\ alle Organe, Strukturen und auch sonst irgendwie offiziell aktiven Studierenden bilden die „organisierte“ StuVe – im Ggs. zur gesamten Studierendenschaft.}


\section*{Zu diesem Heft}

%\vspace{1em}
%\textbf{Zu diesem Handbuch}
%\vspace{0.3em}

Für alle in der StuVe Aktiven und insbesondere die Mitglieder der StuVe-Gremien sind die hier die wichtigsten Texte zusammengefasst, die die Grundlage für die Organisation und Arbeit der StuVe bilden. Ergänzend zu diesem Handbuch soll demnächst noch ein zweites Handbuch mit Hilfen für's operative Tagesgeschäft verfügbar sein.
Es handelt sich hier nicht um ein am Stück geschriebenes Werk, sondern fast ausschließlich um eine Zusammenstellung verschiedener Texte.
Einerseits sind einfache Hilfestellungen, wie z.B.\ die Infoblätter oder Grafiken, andererseits aber auch für alle Studierenden und insbesondere die gewählten Vertreter und Aktiven rechtlich verbindliche Texte, wie z.B.\ die Organisationssatzung oder die Finanzordnung, enthalten.

Da sich die Dokumente bisweilen ändern wird auch dieses Heft immer wieder aktualisiert werden, zum Vergleich verschiedener Ausgaben dient das Datum auf der Titelseite.
Manche der enthaltenen Dokumente haben eine eigene Seitennummerierung – im gesamten Heft ist jedoch jeweils am äußeren Rand die Bezeichnungen des aktuellen Teils und eine durchgehende Seitennummern grau hinterlegt abgedruckt. Auf diese bezieht sich das folgende Inhaltsverzeichnis.


\section*{Inhalt}

%\vspace{1em}
%\textbf{Inhalt}
%\vspace{0.3em}

\toccontents

\null
\textbf{Nicht enthalten:}
\begin{itemize}
	\item Wahlordnung
	\item Landeshoschulgesetz Baden-Württemberg (LHG): Informationen hierzu jedoch unter Rechtliche Grundlagen und Rahmenbedingungen für die neue Studierendenvertretung.
\end{itemize}

%\clearpage

% !TeX root = ../stuve-handbuch-latex.tex
% !TeX encoding = UTF-8
% !TeX spellcheck = de_DE
% !TeX program = pdflatex


% % % % ALTERNATIVE ;-)
% wurde aber rausgelassen, da evtl. unverständlich und demotivierend

%\vfill
%
%\begin{figure}[h]
%	\centering
%	\hspace{1.5cm}\includegraphics[width = 0.6\textwidth, keepaspectratio]
%	{./grafiken/suchen-lesen-verstehen-umsetzen.jpg}
%	%	{./grafiken/suchen-lesen-verstehen-umsetzen-original-cut.jpg}
%\end{figure}
%
%\begin{minipage}{0.8\textwidth}
%\textbf{Verbitterte alte Männer}\\\textit{2014, Filzstift auf Flipchartpapier, Text: Stefan T. Kaufmann, Umrahmung: Tobias Scheinert, Digitalisat: Simon B. Lüke %(dabei später Umrahmung entfernt.)}
%\end{minipage}
%
%\vfill


\null
\clearpage
\null
\vfill
%\vspace{2cm}
Zu unglaublich vielen Themen der StuVe wurde \textit{irgendwann früher}{\textsuperscript\texttrademark} schonmal \textit{irgendwas}{\textsuperscript\textcopyright} gemacht, gelöst, konzipiert, überlegt und oft auch aufgeschrieben. Vielleicht gibt es nur Informationen darüber, wer sich mal damit befasst hatte und ein paar alte Sitzungsprotokolle. Oder es gibt schon eine ausführliche Aufdröselung des Problems und mögliche Ideen zu Lösung. Bei einigen Vorgängen gibt es aber auch sehr ausführliche und ausgetüftelte Anleitungen.

Also muss man bei vielen Themen nicht von ganz vorne beginnen und es lohnt sich, sich erstmal umzuschauen -- z.\,B.~indem man kurz bei denen nachfragt, die schon ein paar Jahre dabei sind. Auf jeden Fall lohnt es sich aber in der Dateiablage und vor allem \textbf{im StuVe-Wiki zu suchen}. Das StuVe-Wiki ist nicht perfekt organisiert oder sofort zu durchschauen\footnote{Außerdem ist es oft ungünstig, dass  manche Informationen über mehrere verschiedenen Wikis verteilt wurden.}, aber das macht nichts denn es gibt eine Suchfunktion, mit der man schnell zu einem Ergebnis kommt\footnote{Nicht vergessen nicht nur nach den Seitenüberschriften, sondern auch im „Text“ suchen zu lassen ;-)}.

Nutzt diese Möglichkeiten und erspart euch viel unnötige Arbeit. Findet Anknüpfungspunkte, findet heraus welche anderen Stellen (in der StuVe, an der Uni oder sonstwo) schon Erfahrungen oder gar Kompetenzen in dem Bereich, in dem ihr was erreichen wollt, haben.
\vfill
\vfill


\clearpage